\clearpage
%\fancypagestyle{plain}{\fancyhead{}\renewcommand{\headrulewidth}{0pt}}
\renewcommand{\headrulewidth}{0pt}
%\null\vspace{\stretch {1}}
\begin{center}
	\huge{\textbf{Sommario}}
\end{center}
\null\vspace{\stretch {1}}

\begin{spacing}{1.25}
	
Le applicazioni dell'Internet of Things (IoT) richiedono spesso una notevole larghezza di banda, bassa latenza e performance affidabili, e al tempo stesso devono rispettare requisiti normativi e di conformità, motivo per cui il Cloud Computing non risulta adatto in questi particolari casi applicativi.

Per ovviare ai problemi sopracitati, negli ultimi anni si sta affermando un nuovo approccio, l'Edge Computing: un’architettura distribuita di micro data center, ciascuno in grado di immagazzinare ed elaborare i dati a livello locale e, in seguito, trasmetterli ad un data center centralizzato o a un database su Cloud.

Edge Engine nasce allo scopo di realizzare un motore il più generico possibile e slegato dall'hardware, tale da raccogliere dati provenienti dai dispositivi ad esso collegati, elaborarli e inviarli su Cloud.

In questo specifico caso si tratterà lo sviluppo di Edge Engine per dispositivi di tipo PC (Windows/ Linux/ MacOS). Il linguaggio di programmazione utilizzato sarà il C++, con l'intento di ottenere un prodotto finale multipiattaforma, caratteristica concorde con i requisiti preposti di genericità e indipendenza dall'hardware.

Una volta completato lo sviluppo del sistema, ne verranno testate le potenzialità prima in un contesto reale, per poi passare ad un ambito virtuale. Nel primo caso si utilizzerà l'engine per il trattamento di dati provenienti da PC (info su RAM e ROM). Nel secondo invece, verrà creato un componente su Unity3D in grado di usufruire dei servizi offerti da Edge Engine, ma applicati a dati ottenuti dalla scena di gioco.

In ultimo, al fine di ottenere un resoconto riguardo l'effettiva efficacia del sistema, oltre che possibili spunti su eventuali criticità, verrà illustrato un esempio di utilizzo della libreria, in ambiente Arduino, da parte di due tesisti triennali.
\end{spacing}
\null\vspace{\stretch {2}}
\clearpage
\renewcommand{\headrulewidth}{0.5pt}

