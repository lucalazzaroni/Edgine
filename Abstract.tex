\null\vspace{\stretch {1}}
\begin{center}
	\huge{\textbf{Abstract}}
\end{center}
Le applicazioni dell'Internet of Things (IoT) richiedono spesso una notevole larghezza di banda, bassa latenza e performance affidabili, e al tempo stesso devono rispettare requisiti normativi e di conformità, motivo per cui il Cloud Computing non risulta adatto in questi particolari casi applicativi.\\
Per ovviare ai problemi sopracitati, negli ultimi anni si sta affermando un nuovo approccio, l'Edge Computing: un’architettura distribuita di micro data center, ciascuno in grado di immagazzinare ed elaborare i dati a livello locale e in seguito trasmetterli ad un data center centralizzato o a un database su Cloud.\\
Edge Engine nasce allo scopo di realizzare un motore il più generico possibile e slegato dall'hardware, tale da raccogliere dati provenienti dai dispositivi ad esso collegati, elaborarli e inviarli su Cloud.\\
In questo specifico caso si tratterà lo sviluppo di Edge Engine per dispositivi fissi. Il linguaggio di programmazione utilizzato sarà il C++ con l'intento di ottenere un prodotto finale multipiattaforma, caratteristica concorde con i requisiti preposti di genericità e indipendenza dall'hardware.\\
DA COMPLETARE CON PARTE UNITY
\null\vspace{\stretch {2}}


